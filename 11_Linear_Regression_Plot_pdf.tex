\documentclass[]{article}
\usepackage{lmodern}
\usepackage{amssymb,amsmath}
\usepackage{ifxetex,ifluatex}
\usepackage{fixltx2e} % provides \textsubscript
\ifnum 0\ifxetex 1\fi\ifluatex 1\fi=0 % if pdftex
  \usepackage[T1]{fontenc}
  \usepackage[utf8]{inputenc}
\else % if luatex or xelatex
  \ifxetex
    \usepackage{mathspec}
  \else
    \usepackage{fontspec}
  \fi
  \defaultfontfeatures{Ligatures=TeX,Scale=MatchLowercase}
  \newcommand{\euro}{€}
\fi
% use upquote if available, for straight quotes in verbatim environments
\IfFileExists{upquote.sty}{\usepackage{upquote}}{}
% use microtype if available
\IfFileExists{microtype.sty}{%
\usepackage{microtype}
\UseMicrotypeSet[protrusion]{basicmath} % disable protrusion for tt fonts
}{}
\usepackage[margin=1in]{geometry}
\usepackage{hyperref}
\PassOptionsToPackage{usenames,dvipsnames}{color} % color is loaded by hyperref
\hypersetup{unicode=true,
            pdftitle={Creating plots in R using ggplot2 - part 11: linear regression plot},
            pdfauthor={Jodie Burchell; Mauricio Vargas Sepúlveda},
            pdfborder={0 0 0},
            breaklinks=true}
\urlstyle{same}  % don't use monospace font for urls
\usepackage{color}
\usepackage{fancyvrb}
\newcommand{\VerbBar}{|}
\newcommand{\VERB}{\Verb[commandchars=\\\{\}]}
\DefineVerbatimEnvironment{Highlighting}{Verbatim}{commandchars=\\\{\}}
% Add ',fontsize=\small' for more characters per line
\usepackage{framed}
\definecolor{shadecolor}{RGB}{248,248,248}
\newenvironment{Shaded}{\begin{snugshade}}{\end{snugshade}}
\newcommand{\KeywordTok}[1]{\textcolor[rgb]{0.13,0.29,0.53}{\textbf{{#1}}}}
\newcommand{\DataTypeTok}[1]{\textcolor[rgb]{0.13,0.29,0.53}{{#1}}}
\newcommand{\DecValTok}[1]{\textcolor[rgb]{0.00,0.00,0.81}{{#1}}}
\newcommand{\BaseNTok}[1]{\textcolor[rgb]{0.00,0.00,0.81}{{#1}}}
\newcommand{\FloatTok}[1]{\textcolor[rgb]{0.00,0.00,0.81}{{#1}}}
\newcommand{\ConstantTok}[1]{\textcolor[rgb]{0.00,0.00,0.00}{{#1}}}
\newcommand{\CharTok}[1]{\textcolor[rgb]{0.31,0.60,0.02}{{#1}}}
\newcommand{\SpecialCharTok}[1]{\textcolor[rgb]{0.00,0.00,0.00}{{#1}}}
\newcommand{\StringTok}[1]{\textcolor[rgb]{0.31,0.60,0.02}{{#1}}}
\newcommand{\VerbatimStringTok}[1]{\textcolor[rgb]{0.31,0.60,0.02}{{#1}}}
\newcommand{\SpecialStringTok}[1]{\textcolor[rgb]{0.31,0.60,0.02}{{#1}}}
\newcommand{\ImportTok}[1]{{#1}}
\newcommand{\CommentTok}[1]{\textcolor[rgb]{0.56,0.35,0.01}{\textit{{#1}}}}
\newcommand{\DocumentationTok}[1]{\textcolor[rgb]{0.56,0.35,0.01}{\textbf{\textit{{#1}}}}}
\newcommand{\AnnotationTok}[1]{\textcolor[rgb]{0.56,0.35,0.01}{\textbf{\textit{{#1}}}}}
\newcommand{\CommentVarTok}[1]{\textcolor[rgb]{0.56,0.35,0.01}{\textbf{\textit{{#1}}}}}
\newcommand{\OtherTok}[1]{\textcolor[rgb]{0.56,0.35,0.01}{{#1}}}
\newcommand{\FunctionTok}[1]{\textcolor[rgb]{0.00,0.00,0.00}{{#1}}}
\newcommand{\VariableTok}[1]{\textcolor[rgb]{0.00,0.00,0.00}{{#1}}}
\newcommand{\ControlFlowTok}[1]{\textcolor[rgb]{0.13,0.29,0.53}{\textbf{{#1}}}}
\newcommand{\OperatorTok}[1]{\textcolor[rgb]{0.81,0.36,0.00}{\textbf{{#1}}}}
\newcommand{\BuiltInTok}[1]{{#1}}
\newcommand{\ExtensionTok}[1]{{#1}}
\newcommand{\PreprocessorTok}[1]{\textcolor[rgb]{0.56,0.35,0.01}{\textit{{#1}}}}
\newcommand{\AttributeTok}[1]{\textcolor[rgb]{0.77,0.63,0.00}{{#1}}}
\newcommand{\RegionMarkerTok}[1]{{#1}}
\newcommand{\InformationTok}[1]{\textcolor[rgb]{0.56,0.35,0.01}{\textbf{\textit{{#1}}}}}
\newcommand{\WarningTok}[1]{\textcolor[rgb]{0.56,0.35,0.01}{\textbf{\textit{{#1}}}}}
\newcommand{\AlertTok}[1]{\textcolor[rgb]{0.94,0.16,0.16}{{#1}}}
\newcommand{\ErrorTok}[1]{\textcolor[rgb]{0.64,0.00,0.00}{\textbf{{#1}}}}
\newcommand{\NormalTok}[1]{{#1}}
\usepackage{graphicx,grffile}
\makeatletter
\def\maxwidth{\ifdim\Gin@nat@width>\linewidth\linewidth\else\Gin@nat@width\fi}
\def\maxheight{\ifdim\Gin@nat@height>\textheight\textheight\else\Gin@nat@height\fi}
\makeatother
% Scale images if necessary, so that they will not overflow the page
% margins by default, and it is still possible to overwrite the defaults
% using explicit options in \includegraphics[width, height, ...]{}
\setkeys{Gin}{width=\maxwidth,height=\maxheight,keepaspectratio}
\setlength{\parindent}{0pt}
\setlength{\parskip}{6pt plus 2pt minus 1pt}
\setlength{\emergencystretch}{3em}  % prevent overfull lines
\providecommand{\tightlist}{%
  \setlength{\itemsep}{0pt}\setlength{\parskip}{0pt}}
\setcounter{secnumdepth}{0}

%%% Use protect on footnotes to avoid problems with footnotes in titles
\let\rmarkdownfootnote\footnote%
\def\footnote{\protect\rmarkdownfootnote}

%%% Change title format to be more compact
\usepackage{titling}

% Create subtitle command for use in maketitle
\newcommand{\subtitle}[1]{
  \posttitle{
    \begin{center}\large#1\end{center}
    }
}

\setlength{\droptitle}{-2em}
  \title{Creating plots in R using ggplot2 - part 11: linear regression plot}
  \pretitle{\vspace{\droptitle}\centering\huge}
  \posttitle{\par}
  \author{Jodie Burchell \\ Mauricio Vargas Sepúlveda}
  \preauthor{\centering\large\emph}
  \postauthor{\par}
  \predate{\centering\large\emph}
  \postdate{\par}
  \date{2016-05-13}



% Redefines (sub)paragraphs to behave more like sections
\ifx\paragraph\undefined\else
\let\oldparagraph\paragraph
\renewcommand{\paragraph}[1]{\oldparagraph{#1}\mbox{}}
\fi
\ifx\subparagraph\undefined\else
\let\oldsubparagraph\subparagraph
\renewcommand{\subparagraph}[1]{\oldsubparagraph{#1}\mbox{}}
\fi

\begin{document}
\maketitle

{
\setcounter{tocdepth}{2}
\tableofcontents
}
This is the eleventh tutorial in a series on using \texttt{ggplot2} I am
creating with \href{http://pachamaltese.github.io/}{Mauricio Vargas
Sepúlveda}. In this tutorial we will demonstrate some of the many
options the \texttt{ggplot2} package.

This post will be much more than showing you how to create regression
plots. Here we are extracting, cleaning and processing financial data
from \href{https://www.quandl.com/}{Quandl}.

Before going ahead, we strongly suggest to create a
\href{https://www.quandl.com/}{Quandl} account in order to obtain an
\href{https://www.quandl.com/account/api}{API key} that allows you to
download data without restrictions. It is even possible to log into
Quandl using Github or Linkedin accounts. Quandl's website has complete
instruction and they have an API that is 100\% R compatible.

The goal is to estimate the CAPM model
\(R_i = R_f + \beta_i [R_m - R_f] + e_i\) where \(R_i\) is the return of
an asset, \(R_f\) is the risk-free return (e.g.~US Treasury Bonds),
\(R_m\) is the return of the market portfolio (e.g.~NYSE) and
\(\beta_i\) is a measure of risk relative to the market (e.g.
\(\beta_i = 1\) means that asset is exactly as risky as the market
portfolio). More on the CAPM model can be read
\href{http://people.stern.nyu.edu/ashapiro/courses/B01.231103/FFL09.pdf}{here},
but in this tutorial we will focus on plotting.

In this tutorial, we will work towards creating the trend line and
diagnostics plots below. We will take you from a basic regression plot
and explain all the customisations we add to the code step-by-step.

\begin{center}\includegraphics{11_Linear_Regression_Plot_pdf/lr_final-1} \end{center}

\begin{center}\includegraphics{11_Linear_Regression_Plot_pdf/lr_final-2} \end{center}

The first thing to do is download and load in the libraries and the data
of the monthly price of Hang Seng Index and Cheung Kong Holdings Hong
Kong from 2015-03-01 to 2016-04-01.

\begin{Shaded}
\begin{Highlighting}[]
\KeywordTok{library}\NormalTok{(ggplot2)}
\KeywordTok{library}\NormalTok{(ggthemes)}
\KeywordTok{library}\NormalTok{(grid) }
\KeywordTok{library}\NormalTok{(ggfortify)}
\KeywordTok{library}\NormalTok{(Quandl)}
\KeywordTok{Quandl.api_key}\NormalTok{(}\StringTok{"XXX"}\NormalTok{)}

\NormalTok{hsi.df <-}\StringTok{ }\KeywordTok{Quandl}\NormalTok{(}\StringTok{"YAHOO/INDEX_HSI"}\NormalTok{, }\DataTypeTok{start_date=}\StringTok{"2015-03-01"}\NormalTok{, }\DataTypeTok{end_date=}\StringTok{"2016-04-01"}\NormalTok{,}
    \DataTypeTok{collapse=}\StringTok{"monthly"}\NormalTok{, }\DataTypeTok{type =} \StringTok{"raw"}\NormalTok{)}

\NormalTok{ckh.df <-}\StringTok{ }\KeywordTok{Quandl}\NormalTok{(}\StringTok{"YAHOO/HK_0001"}\NormalTok{, }\DataTypeTok{start_date=}\StringTok{"2015-03-01"}\NormalTok{, }
    \DataTypeTok{end_date=}\StringTok{"2016-04-01"}\NormalTok{, }\DataTypeTok{collapse=}\StringTok{"monthly"}\NormalTok{, }\DataTypeTok{type =} \StringTok{"raw"}\NormalTok{)}

\KeywordTok{saveRDS}\NormalTok{(hsi.df, }\StringTok{"hsi.rds"}\NormalTok{); }\KeywordTok{saveRDS}\NormalTok{(ckh.df,}\StringTok{"ckh.rds"}\NormalTok{)}
\end{Highlighting}
\end{Shaded}

Before calculating return as
\(R_i = \displaystyle \frac{P_t - P_{t-1}}{P_t}\) we need to order HSI
and CKH data by dates and in decreasing order.

\begin{Shaded}
\begin{Highlighting}[]
\NormalTok{hsi.df <-}\StringTok{ }\KeywordTok{readRDS}\NormalTok{(}\StringTok{"hsi.rds"}\NormalTok{)}
\KeywordTok{colnames}\NormalTok{(hsi.df)[}\DecValTok{7}\NormalTok{] <-}\StringTok{ "Adjusted.Close"}
\NormalTok{hsi.df <-}\StringTok{ }\NormalTok{hsi.df[}\KeywordTok{order}\NormalTok{(}\KeywordTok{as.Date}\NormalTok{(hsi.df$Date)),]}
\end{Highlighting}
\end{Shaded}

With ordered dates it is possible to obtain the correct return for each
month.

\begin{Shaded}
\begin{Highlighting}[]
\NormalTok{hsi.Adjusted.Close <-}\StringTok{ }\NormalTok{hsi.df$Adjusted.Close}
\NormalTok{hsi.Return <-}\StringTok{ }\KeywordTok{diff}\NormalTok{(hsi.Adjusted.Close)/hsi.Adjusted.Close[-}\KeywordTok{length}\NormalTok{(hsi.Adjusted.Close)]}
\NormalTok{hsi.Return <-}\StringTok{ }\KeywordTok{c}\NormalTok{(}\OtherTok{NA}\NormalTok{,hsi.Return)}
\NormalTok{hsi.df$Return <-}\StringTok{ }\NormalTok{hsi.Return}
\NormalTok{hsi.df <-}\StringTok{ }\KeywordTok{na.omit}\NormalTok{(hsi.df)}
\NormalTok{hsi.Return <-}\StringTok{ }\NormalTok{hsi.df[,}\KeywordTok{c}\NormalTok{(}\StringTok{"Date"}\NormalTok{,}\StringTok{"Return"}\NormalTok{)]}

\NormalTok{ckh.df <-}\StringTok{ }\KeywordTok{readRDS}\NormalTok{(}\StringTok{"ckh.rds"}\NormalTok{)}
\KeywordTok{colnames}\NormalTok{(ckh.df)[}\DecValTok{7}\NormalTok{] <-}\StringTok{ "Adjusted.Close"}
\NormalTok{ckh.df <-}\StringTok{ }\NormalTok{ckh.df[}\KeywordTok{order}\NormalTok{(}\KeywordTok{as.Date}\NormalTok{(ckh.df$Date)),]}
\NormalTok{ckh.Adjusted.Close <-}\StringTok{ }\NormalTok{ckh.df$Adjusted.Close}
\NormalTok{ckh.Return <-}\StringTok{ }\KeywordTok{diff}\NormalTok{(ckh.Adjusted.Close)/ckh.Adjusted.Close[-}\KeywordTok{length}\NormalTok{(ckh.Adjusted.Close)]}
\NormalTok{ckh.Return <-}\StringTok{ }\KeywordTok{c}\NormalTok{(}\OtherTok{NA}\NormalTok{,ckh.Return)}
\NormalTok{ckh.df <-}\StringTok{ }\KeywordTok{na.omit}\NormalTok{(ckh.df)}
\NormalTok{ckh.df$Return <-}\StringTok{ }\NormalTok{ckh.Return}
\NormalTok{ckh.Return <-}\StringTok{ }\NormalTok{ckh.df[,}\KeywordTok{c}\NormalTok{(}\StringTok{"Date"}\NormalTok{,}\StringTok{"Return"}\NormalTok{)]}
\end{Highlighting}
\end{Shaded}

The returns can be arranged in one data frame before doing plots and
regression.

\begin{Shaded}
\begin{Highlighting}[]
\NormalTok{hsi.ckh.returns <-}\StringTok{ }\KeywordTok{merge}\NormalTok{(hsi.Return, ckh.Return, }\DataTypeTok{by=}\StringTok{'Date'}\NormalTok{)}
\NormalTok{hsi.ckh.returns <-}\StringTok{ }\KeywordTok{na.omit}\NormalTok{(hsi.ckh.returns)}
\KeywordTok{colnames}\NormalTok{(hsi.ckh.returns) <-}\StringTok{ }\KeywordTok{c}\NormalTok{(}\StringTok{"Date"}\NormalTok{,}\StringTok{"hsi.Return"}\NormalTok{,}\StringTok{"ckh.Return"}\NormalTok{)}
\end{Highlighting}
\end{Shaded}

Using
\href{http://pages.stern.nyu.edu/~adamodar/New_Home_Page/datafile/ctryprem.html}{Damodaran}
and
\href{http://www.bloomberg.com/markets/rates-bonds/government-bonds/us}{Bloomberg}
data we can work with an estimate of HSI risk premium over risk-free
rate.

\begin{Shaded}
\begin{Highlighting}[]
\NormalTok{usa.risk.free <-}\StringTok{ }\FloatTok{0.3}\NormalTok{/}\DecValTok{100}
\NormalTok{hsi.risk.premium <-}\StringTok{ }\FloatTok{0.6}\NormalTok{/}\DecValTok{100}
\end{Highlighting}
\end{Shaded}

\section{Trend line plot}\label{trend-line-plot}

\subsection{Basic trend line plot}\label{basic-trend-line-plot}

Now we can fit a linear regression. One interesting thing is that in
CAPM context the regression line slope can be calculated as
\(\beta_i = \displaystyle \frac{\sigma_{i,m}}{\sigma_m^2}\).

\begin{Shaded}
\begin{Highlighting}[]
\NormalTok{hsi.ckh.returns$hsi.Risk.free <-}\StringTok{ }\NormalTok{usa.risk.free +}\StringTok{ }\NormalTok{hsi.risk.premium}
\NormalTok{hsi.ckh.returns$hsi.Risk.premium <-}\StringTok{ }\NormalTok{hsi.ckh.returns$hsi.Return -}\StringTok{ }\NormalTok{hsi.ckh.returns$hsi.Risk.free}
\NormalTok{hsi.Return.vector <-}\StringTok{ }\KeywordTok{as.vector}\NormalTok{(hsi.ckh.returns$hsi.Return)}
\NormalTok{ckh.Return.vector <-}\StringTok{ }\KeywordTok{as.vector}\NormalTok{(hsi.ckh.returns$ckh.Return)}
\NormalTok{cov.hsi.ckh <-}\StringTok{ }\KeywordTok{cov}\NormalTok{(ckh.Return.vector, hsi.Return.vector)}
\NormalTok{var.hk <-}\StringTok{ }\KeywordTok{var}\NormalTok{(hsi.Return.vector)}
\NormalTok{capm_beta =}\StringTok{ }\NormalTok{cov.hsi.ckh/var.hk}

\NormalTok{fit <-}\StringTok{ }\KeywordTok{lm}\NormalTok{(ckh.Return ~}\StringTok{ }\NormalTok{hsi.Risk.premium, }\DataTypeTok{data =} \NormalTok{hsi.ckh.returns) }
\KeywordTok{summary}\NormalTok{(fit)}
\end{Highlighting}
\end{Shaded}

\begin{verbatim}

Call:
lm(formula = ckh.Return ~ hsi.Risk.premium, data = hsi.ckh.returns)

Residuals:
      Min        1Q    Median        3Q       Max 
-0.031033 -0.022800  0.001032  0.010137  0.050709 

Coefficients:
                 Estimate Std. Error t value Pr(>|t|)    
(Intercept)      0.007142   0.007437   0.960    0.357    
hsi.Risk.premium 0.671372   0.101209   6.634 3.69e-05 ***
---
Signif. codes:  0 '***' 0.001 '**' 0.01 '*' 0.05 '.' 0.1 ' ' 1

Residual standard error: 0.02565 on 11 degrees of freedom
Multiple R-squared:    0.8, Adjusted R-squared:  0.7818 
F-statistic:    44 on 1 and 11 DF,  p-value: 3.692e-05
\end{verbatim}

Up to this point we have all we need to plot regressions. We will start
with a basic regression plot.

\begin{Shaded}
\begin{Highlighting}[]
\NormalTok{p11 <-}\StringTok{ }\KeywordTok{ggplot}\NormalTok{(hsi.ckh.returns, }\KeywordTok{aes}\NormalTok{(}\DataTypeTok{x=}\NormalTok{hsi.Risk.premium, }\DataTypeTok{y=}\NormalTok{ckh.Return)) +}\StringTok{ }\KeywordTok{geom_point}\NormalTok{(}\DataTypeTok{shape=}\DecValTok{1}\NormalTok{) +}\StringTok{ }\KeywordTok{geom_smooth}\NormalTok{(}\DataTypeTok{method=}\NormalTok{lm) }
\NormalTok{p11}
\end{Highlighting}
\end{Shaded}

\begin{center}\includegraphics{11_Linear_Regression_Plot_pdf/lr_6-1} \end{center}

\texttt{geom\_smooth} can be customized, for example, not to include the
confidence region

\begin{Shaded}
\begin{Highlighting}[]
\NormalTok{p11 <-}\StringTok{ }\KeywordTok{ggplot}\NormalTok{(hsi.ckh.returns, }\KeywordTok{aes}\NormalTok{(}\DataTypeTok{x=}\NormalTok{hsi.Risk.premium, }\DataTypeTok{y=}\NormalTok{ckh.Return)) +}\StringTok{ }\KeywordTok{geom_point}\NormalTok{(}\DataTypeTok{shape=}\DecValTok{1}\NormalTok{) +}\StringTok{ }\KeywordTok{geom_smooth}\NormalTok{(}\DataTypeTok{method=}\NormalTok{lm, }\DataTypeTok{se=}\OtherTok{FALSE}\NormalTok{) }
\NormalTok{p11}
\end{Highlighting}
\end{Shaded}

\begin{center}\includegraphics{11_Linear_Regression_Plot_pdf/lr_7-1} \end{center}

Before continuing it is a good idea to fix the axis labels and add a
title.

\subsection{Customising axis labels}\label{customising-axis-labels}

We can change the text of the axis labels using the
\texttt{scale\_x\_continuous} and \texttt{scale\_y\_continuous} options,
with the names passed as a string to the \texttt{name} arguments in
each.

\begin{Shaded}
\begin{Highlighting}[]
\NormalTok{p11 <-}\StringTok{ }\NormalTok{p11 +}\StringTok{ }\KeywordTok{scale_x_continuous}\NormalTok{(}\DataTypeTok{name =} \StringTok{"HSI risk premium"}\NormalTok{) +}\StringTok{ }
\StringTok{  }\KeywordTok{scale_y_continuous}\NormalTok{(}\DataTypeTok{name =} \StringTok{"CKH return"}\NormalTok{)}
\NormalTok{p11}
\end{Highlighting}
\end{Shaded}

\begin{center}\includegraphics{11_Linear_Regression_Plot_pdf/lr_8-1} \end{center}

\subsection{Adding a title}\label{adding-a-title}

Similarly, we can add a title using the \texttt{ggtitle} option.

\begin{Shaded}
\begin{Highlighting}[]
\NormalTok{p11 <-}\StringTok{ }\NormalTok{p11 +}\StringTok{ }\KeywordTok{ggtitle}\NormalTok{(}\StringTok{"CKH regression line"}\NormalTok{)}
\NormalTok{p11}
\end{Highlighting}
\end{Shaded}

\begin{center}\includegraphics{11_Linear_Regression_Plot_pdf/lr_9-1} \end{center}

\subsection{Including regression
coefficients}\label{including-regression-coefficients}

We can also include more information about the regression line itself.
It would be interesting to show \(R^2\) and regression coefficients
within the plot.

\begin{Shaded}
\begin{Highlighting}[]
\NormalTok{p11 <-}\StringTok{ }\NormalTok{p11 +}\StringTok{ }\KeywordTok{annotate}\NormalTok{(}\StringTok{"text"}\NormalTok{, }\DataTypeTok{x=}\FloatTok{0.1}\NormalTok{, }\DataTypeTok{y=}\NormalTok{-}\FloatTok{0.05}\NormalTok{, }\DataTypeTok{label =} \StringTok{"R^2=0.78"}\NormalTok{) +}\StringTok{ }
\StringTok{  }\KeywordTok{annotate}\NormalTok{(}\StringTok{"text"}\NormalTok{, }\DataTypeTok{x=}\FloatTok{0.1}\NormalTok{, }\DataTypeTok{y=}\NormalTok{-}\FloatTok{0.06}\NormalTok{, }\DataTypeTok{label =} \StringTok{"alpha=0.00"}\NormalTok{) +}\StringTok{ }
\StringTok{  }\KeywordTok{annotate}\NormalTok{(}\StringTok{"text"}\NormalTok{, }\DataTypeTok{x=}\FloatTok{0.1}\NormalTok{, }\DataTypeTok{y=}\NormalTok{-}\FloatTok{0.07}\NormalTok{, }\DataTypeTok{label =} \StringTok{"beta=0.67"}\NormalTok{)}
\NormalTok{p11}
\end{Highlighting}
\end{Shaded}

\begin{center}\includegraphics{11_Linear_Regression_Plot_pdf/lr_10-1} \end{center}

Another option would be to add greek letters and exponents.

\begin{Shaded}
\begin{Highlighting}[]
\NormalTok{p11 <-}\StringTok{ }\KeywordTok{ggplot}\NormalTok{(hsi.ckh.returns, }\KeywordTok{aes}\NormalTok{(}\DataTypeTok{x=}\NormalTok{hsi.Risk.premium, }\DataTypeTok{y=}\NormalTok{ckh.Return)) +}\StringTok{ }\KeywordTok{geom_point}\NormalTok{(}\DataTypeTok{shape=}\DecValTok{1}\NormalTok{) +}\StringTok{ }
\StringTok{  }\KeywordTok{geom_smooth}\NormalTok{(}\DataTypeTok{method=}\NormalTok{lm, }\DataTypeTok{se=}\OtherTok{FALSE}\NormalTok{) +}\StringTok{ }\KeywordTok{ggtitle}\NormalTok{(}\StringTok{"CKH regression line"}\NormalTok{) +}
\StringTok{  }\KeywordTok{scale_x_continuous}\NormalTok{(}\DataTypeTok{name =} \StringTok{"HSI risk premium"}\NormalTok{) +}
\StringTok{  }\KeywordTok{scale_y_continuous}\NormalTok{(}\DataTypeTok{name =} \StringTok{"CKH return"}\NormalTok{) +}\StringTok{ }
\StringTok{  }\KeywordTok{annotate}\NormalTok{(}\StringTok{"text"}\NormalTok{, }\DataTypeTok{x=}\FloatTok{0.1}\NormalTok{, }\DataTypeTok{y=}\NormalTok{-}\FloatTok{0.05}\NormalTok{, }\DataTypeTok{label =} \StringTok{"R^2 == 0.78"}\NormalTok{, }\DataTypeTok{parse=}\NormalTok{T) +}\StringTok{ }
\StringTok{  }\KeywordTok{annotate}\NormalTok{(}\StringTok{"text"}\NormalTok{, }\DataTypeTok{x=}\FloatTok{0.1}\NormalTok{, }\DataTypeTok{y=}\NormalTok{-}\FloatTok{0.06}\NormalTok{, }\DataTypeTok{label =} \StringTok{"alpha == 0.00"}\NormalTok{, }\DataTypeTok{parse=}\NormalTok{T) +}\StringTok{ }
\StringTok{  }\KeywordTok{annotate}\NormalTok{(}\StringTok{"text"}\NormalTok{, }\DataTypeTok{x=}\FloatTok{0.1}\NormalTok{, }\DataTypeTok{y=}\NormalTok{-}\FloatTok{0.07}\NormalTok{, }\DataTypeTok{label =} \StringTok{"beta == 0.67"}\NormalTok{, }\DataTypeTok{parse=}\NormalTok{T)}
\NormalTok{p11}
\end{Highlighting}
\end{Shaded}

\begin{center}\includegraphics{11_Linear_Regression_Plot_pdf/lr_11-1} \end{center}

To make the coefficients more clear we will add some elements to
increase visibility.

\begin{Shaded}
\begin{Highlighting}[]
\NormalTok{p11 <-}\StringTok{ }\KeywordTok{ggplot}\NormalTok{(hsi.ckh.returns, }\KeywordTok{aes}\NormalTok{(}\DataTypeTok{x=}\NormalTok{hsi.Risk.premium, }\DataTypeTok{y=}\NormalTok{ckh.Return)) +}\StringTok{ }
\StringTok{  }\KeywordTok{geom_point}\NormalTok{(}\DataTypeTok{shape=}\DecValTok{1}\NormalTok{) +}\StringTok{ }\KeywordTok{geom_smooth}\NormalTok{(}\DataTypeTok{method=}\NormalTok{lm, }\DataTypeTok{se=}\OtherTok{FALSE}\NormalTok{) +}
\StringTok{  }\KeywordTok{ggtitle}\NormalTok{(}\StringTok{"CKH regression line"}\NormalTok{) +}
\StringTok{  }\KeywordTok{scale_x_continuous}\NormalTok{(}\DataTypeTok{name =} \StringTok{"HSI risk premium"}\NormalTok{) +}
\StringTok{  }\KeywordTok{scale_y_continuous}\NormalTok{(}\DataTypeTok{name =} \StringTok{"CKH return"}\NormalTok{) +}
\StringTok{  }\KeywordTok{annotate}\NormalTok{(}\StringTok{"rect"}\NormalTok{, }\DataTypeTok{xmin =} \FloatTok{0.075}\NormalTok{, }\DataTypeTok{xmax =} \FloatTok{0.125}\NormalTok{, }\DataTypeTok{ymin =} \NormalTok{-}\FloatTok{0.075}\NormalTok{, }\DataTypeTok{ymax =} \NormalTok{-}\FloatTok{0.045}\NormalTok{, }\DataTypeTok{fill=}\StringTok{"white"}\NormalTok{, }
    \DataTypeTok{colour=}\StringTok{"red"}\NormalTok{) +}\StringTok{ }
\StringTok{  }\KeywordTok{annotate}\NormalTok{(}\StringTok{"text"}\NormalTok{, }\DataTypeTok{x=}\FloatTok{0.1}\NormalTok{, }\DataTypeTok{y=}\NormalTok{-}\FloatTok{0.05}\NormalTok{, }\DataTypeTok{label =} \StringTok{"R^2 == 0.78"}\NormalTok{, }\DataTypeTok{parse=}\NormalTok{T) +}\StringTok{ }\KeywordTok{annotate}\NormalTok{(}\StringTok{"text"}\NormalTok{, }\DataTypeTok{x=}\FloatTok{0.1}\NormalTok{, }\DataTypeTok{y=}\NormalTok{-}\FloatTok{0.06}\NormalTok{, }
    \DataTypeTok{label =} \StringTok{"alpha == 0.00"}\NormalTok{, }\DataTypeTok{parse=}\NormalTok{T) +}\StringTok{ }
\StringTok{  }\KeywordTok{annotate}\NormalTok{(}\StringTok{"text"}\NormalTok{, }\DataTypeTok{x=}\FloatTok{0.1}\NormalTok{, }\DataTypeTok{y=}\NormalTok{-}\FloatTok{0.07}\NormalTok{, }\DataTypeTok{label =} \StringTok{"beta == 0.67"}\NormalTok{, }\DataTypeTok{parse=}\NormalTok{T)}
\NormalTok{p11}
\end{Highlighting}
\end{Shaded}

\begin{center}\includegraphics{11_Linear_Regression_Plot_pdf/lr_12-1} \end{center}

Another customization could be to show the trend line using rounded
digits (or even significant digits) from regression coefficients. This
requires us to write a function and is not as easy to obtain as the last
plot.

\begin{Shaded}
\begin{Highlighting}[]
\NormalTok{equation =}\StringTok{ }\NormalTok{function(x) \{}
  \NormalTok{lm_coef <-}\StringTok{ }\KeywordTok{list}\NormalTok{(}\DataTypeTok{a =} \KeywordTok{round}\NormalTok{(}\KeywordTok{coef}\NormalTok{(x)[}\DecValTok{1}\NormalTok{], }\DataTypeTok{digits =} \DecValTok{2}\NormalTok{),}
    \DataTypeTok{b =} \KeywordTok{round}\NormalTok{(}\KeywordTok{coef}\NormalTok{(x)[}\DecValTok{2}\NormalTok{], }\DataTypeTok{digits =} \DecValTok{2}\NormalTok{),}
    \DataTypeTok{r2 =} \KeywordTok{round}\NormalTok{(}\KeywordTok{summary}\NormalTok{(x)$r.squared, }\DataTypeTok{digits =} \DecValTok{2}\NormalTok{));}
  \NormalTok{lm_eq <-}\StringTok{ }\KeywordTok{substitute}\NormalTok{(}\KeywordTok{italic}\NormalTok{(y) ==}\StringTok{ }\NormalTok{a +}\StringTok{ }\NormalTok{b %.%}\StringTok{ }\KeywordTok{italic}\NormalTok{(x)*}\StringTok{","}\NormalTok{~}\ErrorTok{~}\KeywordTok{italic}\NormalTok{(R)^}\DecValTok{2}\NormalTok{~}\StringTok{"="}\NormalTok{~r2,lm_coef)}
  \KeywordTok{as.character}\NormalTok{(}\KeywordTok{as.expression}\NormalTok{(lm_eq));         }
\NormalTok{\}}

\NormalTok{p11 <-}\StringTok{ }\KeywordTok{ggplot}\NormalTok{(hsi.ckh.returns, }\KeywordTok{aes}\NormalTok{(}\DataTypeTok{x=}\NormalTok{hsi.Risk.premium, }\DataTypeTok{y=}\NormalTok{ckh.Return)) +}\StringTok{ }\KeywordTok{geom_point}\NormalTok{(}\DataTypeTok{shape=}\DecValTok{1}\NormalTok{) +}\StringTok{ }\KeywordTok{geom_smooth}\NormalTok{(}\DataTypeTok{method=}\NormalTok{lm, }\DataTypeTok{se=}\OtherTok{FALSE}\NormalTok{) +}
\StringTok{  }\KeywordTok{ggtitle}\NormalTok{(}\StringTok{"CKH regression line"}\NormalTok{) +}
\StringTok{  }\KeywordTok{scale_x_continuous}\NormalTok{(}\DataTypeTok{name =} \StringTok{"HSI risk premium"}\NormalTok{) +}
\StringTok{  }\KeywordTok{scale_y_continuous}\NormalTok{(}\DataTypeTok{name =} \StringTok{"CKH return"}\NormalTok{) +}
\StringTok{  }\KeywordTok{annotate}\NormalTok{(}\StringTok{"rect"}\NormalTok{, }\DataTypeTok{xmin =} \FloatTok{0.00}\NormalTok{, }\DataTypeTok{xmax =} \FloatTok{0.1}\NormalTok{, }\DataTypeTok{ymin =} \NormalTok{-}\FloatTok{0.056}\NormalTok{, }\DataTypeTok{ymax =} \NormalTok{-}\FloatTok{0.044}\NormalTok{, }\DataTypeTok{fill=}\StringTok{"white"}\NormalTok{, }\DataTypeTok{colour=}\StringTok{"red"}\NormalTok{) +}\StringTok{ }
\StringTok{  }\KeywordTok{annotate}\NormalTok{(}\StringTok{"text"}\NormalTok{, }\DataTypeTok{x =} \FloatTok{0.05}\NormalTok{, }\DataTypeTok{y =} \NormalTok{-}\FloatTok{0.05}\NormalTok{, }\DataTypeTok{label =} \KeywordTok{equation}\NormalTok{(fit), }\DataTypeTok{parse =} \OtherTok{TRUE}\NormalTok{)}
\NormalTok{p11}
\end{Highlighting}
\end{Shaded}

\begin{center}\includegraphics{11_Linear_Regression_Plot_pdf/lr_13-1} \end{center}

\subsection{Using the white theme}\label{using-the-white-theme}

As explained in the previous posts, we can also change the overall look
of the plot using themes. We'll start using a simple theme customisation
by adding \texttt{theme\_bw()}. As you can see, we can further tweak the
graph using the \texttt{theme} option, which we've used so far to change
the legend.

\begin{Shaded}
\begin{Highlighting}[]
\NormalTok{p11 <-}\StringTok{ }\NormalTok{p11 +}\StringTok{ }\KeywordTok{theme_bw}\NormalTok{()}
\NormalTok{p11}
\end{Highlighting}
\end{Shaded}

\begin{center}\includegraphics{11_Linear_Regression_Plot_pdf/lr_14-1} \end{center}

\subsection{Creating an XKCD style
chart}\label{creating-an-xkcd-style-chart}

Of course, you may want to create your own themes as well.
\texttt{ggplot2} allows for a very high degree of customisation,
including allowing you to use imported fonts. Below is an example of a
theme Mauricio was able to create which mimics the visual style of
\href{http://xkcd.com/}{XKCD}. In order to create this chart, you first
need to import the XKCD font, and load it into R using the
\texttt{extrafont} package.

\begin{Shaded}
\begin{Highlighting}[]
\NormalTok{p11 <-}\StringTok{ }\KeywordTok{ggplot}\NormalTok{(hsi.ckh.returns, }\KeywordTok{aes}\NormalTok{(}\DataTypeTok{x=}\NormalTok{hsi.Risk.premium, }\DataTypeTok{y=}\NormalTok{ckh.Return)) +}\StringTok{ }\KeywordTok{geom_point}\NormalTok{(}\DataTypeTok{shape=}\DecValTok{1}\NormalTok{) +}\StringTok{ }\KeywordTok{geom_smooth}\NormalTok{(}\DataTypeTok{method=}\NormalTok{lm, }\DataTypeTok{se=}\OtherTok{FALSE}\NormalTok{) +}
\StringTok{  }\KeywordTok{ggtitle}\NormalTok{(}\StringTok{"CKH regression line"}\NormalTok{) +}
\StringTok{  }\KeywordTok{scale_x_continuous}\NormalTok{(}\DataTypeTok{name =} \StringTok{"HSI risk premium"}\NormalTok{) +}
\StringTok{  }\KeywordTok{scale_y_continuous}\NormalTok{(}\DataTypeTok{name =} \StringTok{"CKH return"}\NormalTok{) +}
\StringTok{  }\KeywordTok{annotate}\NormalTok{(}\StringTok{"rect"}\NormalTok{, }\DataTypeTok{xmin =} \FloatTok{0.00}\NormalTok{, }\DataTypeTok{xmax =} \FloatTok{0.1}\NormalTok{, }\DataTypeTok{ymin =} \NormalTok{-}\FloatTok{0.056}\NormalTok{, }\DataTypeTok{ymax =} \NormalTok{-}\FloatTok{0.044}\NormalTok{, }\DataTypeTok{fill=}\StringTok{"white"}\NormalTok{, }\DataTypeTok{colour=}\StringTok{"red"}\NormalTok{) +}\StringTok{ }
\StringTok{  }\KeywordTok{annotate}\NormalTok{(}\StringTok{"text"}\NormalTok{, }\DataTypeTok{x =} \FloatTok{0.05}\NormalTok{, }\DataTypeTok{y =} \NormalTok{-}\FloatTok{0.05}\NormalTok{, }\DataTypeTok{label =} \KeywordTok{equation}\NormalTok{(fit), }\DataTypeTok{parse =} \OtherTok{TRUE}\NormalTok{) +}\StringTok{ }
\StringTok{  }\KeywordTok{theme}\NormalTok{(}\DataTypeTok{axis.line.x =} \KeywordTok{element_line}\NormalTok{(}\DataTypeTok{size=}\NormalTok{.}\DecValTok{5}\NormalTok{, }\DataTypeTok{colour =} \StringTok{"black"}\NormalTok{), }
    \DataTypeTok{axis.line.y =} \KeywordTok{element_line}\NormalTok{(}\DataTypeTok{size=}\NormalTok{.}\DecValTok{5}\NormalTok{, }\DataTypeTok{colour =} \StringTok{"black"}\NormalTok{),     }
    \DataTypeTok{axis.text.x=}\KeywordTok{element_text}\NormalTok{(}\DataTypeTok{colour=}\StringTok{"black"}\NormalTok{, }\DataTypeTok{size =} \DecValTok{10}\NormalTok{), }
    \DataTypeTok{axis.text.y=}\KeywordTok{element_text}\NormalTok{(}\DataTypeTok{colour=}\StringTok{"black"}\NormalTok{, }\DataTypeTok{size =} \DecValTok{10}\NormalTok{), }
    \DataTypeTok{legend.position=}\StringTok{"bottom"}\NormalTok{, }
    \DataTypeTok{legend.direction=}\StringTok{"horizontal"}\NormalTok{,}
    \DataTypeTok{legend.box =} \StringTok{"horizontal"}\NormalTok{, }
    \DataTypeTok{legend.key =} \KeywordTok{element_blank}\NormalTok{(),}
    \DataTypeTok{panel.grid.major =} \KeywordTok{element_blank}\NormalTok{(),}
    \DataTypeTok{panel.grid.minor =} \KeywordTok{element_blank}\NormalTok{(), }
    \DataTypeTok{panel.border =} \KeywordTok{element_blank}\NormalTok{(),}
    \DataTypeTok{panel.background =} \KeywordTok{element_blank}\NormalTok{(),}
    \DataTypeTok{plot.title=}\KeywordTok{element_text}\NormalTok{(}\DataTypeTok{family=}\StringTok{"xkcd-Regular"}\NormalTok{), }
    \DataTypeTok{text=}\KeywordTok{element_text}\NormalTok{(}\DataTypeTok{family=}\StringTok{"xkcd-Regular"}\NormalTok{)) }
\NormalTok{p11}
\end{Highlighting}
\end{Shaded}

\begin{center}\includegraphics{11_Linear_Regression_Plot_pdf/lr_15-1} \end{center}

\subsection{\texorpdfstring{Using `The Economist'
theme}{Using The Economist theme}}\label{using-the-economist-theme}

There are a wider range of pre-built themes available as part of the
\texttt{ggthemes} package (more information on these
\href{https://cran.r-project.org/web/packages/ggthemes/vignettes/ggthemes.html}{here}).
Below we've applied \texttt{theme\_economist()}, which approximates
graphs in the Economist magazine.

\begin{Shaded}
\begin{Highlighting}[]
\NormalTok{p11 <-}\StringTok{ }\KeywordTok{ggplot}\NormalTok{(hsi.ckh.returns, }\KeywordTok{aes}\NormalTok{(}\DataTypeTok{x=}\NormalTok{hsi.Risk.premium, }\DataTypeTok{y=}\NormalTok{ckh.Return)) +}\StringTok{ }\KeywordTok{geom_point}\NormalTok{(}\DataTypeTok{shape=}\DecValTok{1}\NormalTok{) +}\StringTok{ }\KeywordTok{geom_smooth}\NormalTok{(}\DataTypeTok{method=}\NormalTok{lm, }\DataTypeTok{se=}\OtherTok{FALSE}\NormalTok{) +}
\StringTok{  }\KeywordTok{ggtitle}\NormalTok{(}\StringTok{"CKH regression line"}\NormalTok{) +}
\StringTok{  }\KeywordTok{scale_x_continuous}\NormalTok{(}\DataTypeTok{name =} \StringTok{"HSI risk premium"}\NormalTok{) +}
\StringTok{  }\KeywordTok{scale_y_continuous}\NormalTok{(}\DataTypeTok{name =} \StringTok{"CKH return"}\NormalTok{) +}
\StringTok{  }\KeywordTok{annotate}\NormalTok{(}\StringTok{"rect"}\NormalTok{, }\DataTypeTok{xmin =} \NormalTok{-}\FloatTok{0.002}\NormalTok{, }\DataTypeTok{xmax =} \FloatTok{0.102}\NormalTok{, }\DataTypeTok{ymin =} \NormalTok{-}\FloatTok{0.056}\NormalTok{, }\DataTypeTok{ymax =} \NormalTok{-}\FloatTok{0.044}\NormalTok{, }\DataTypeTok{fill=}\StringTok{"white"}\NormalTok{, }
    \DataTypeTok{colour=}\StringTok{"red"}\NormalTok{) +}\StringTok{ }
\StringTok{  }\KeywordTok{annotate}\NormalTok{(}\StringTok{"text"}\NormalTok{, }\DataTypeTok{x =} \FloatTok{0.05}\NormalTok{, }\DataTypeTok{y =} \NormalTok{-}\FloatTok{0.05}\NormalTok{, }\DataTypeTok{label =} \KeywordTok{equation}\NormalTok{(fit), }\DataTypeTok{parse =} \OtherTok{TRUE}\NormalTok{) +}\StringTok{ }
\StringTok{  }\KeywordTok{theme_economist}\NormalTok{() +}\StringTok{ }\KeywordTok{scale_fill_economist}\NormalTok{() +}
\StringTok{  }\KeywordTok{theme}\NormalTok{(}\DataTypeTok{axis.line.x =} \KeywordTok{element_line}\NormalTok{(}\DataTypeTok{size=}\NormalTok{.}\DecValTok{5}\NormalTok{, }\DataTypeTok{colour =} \StringTok{"black"}\NormalTok{),}
    \DataTypeTok{axis.title =} \KeywordTok{element_text}\NormalTok{(}\DataTypeTok{size =} \DecValTok{12}\NormalTok{),}
    \DataTypeTok{legend.position=}\StringTok{"bottom"}\NormalTok{, }
    \DataTypeTok{legend.direction=}\StringTok{"horizontal"}\NormalTok{,}
    \DataTypeTok{legend.box =} \StringTok{"horizontal"}\NormalTok{, }
    \DataTypeTok{legend.text =} \KeywordTok{element_text}\NormalTok{(}\DataTypeTok{size =} \DecValTok{10}\NormalTok{),}
    \DataTypeTok{text =} \KeywordTok{element_text}\NormalTok{(}\DataTypeTok{family =} \StringTok{"OfficinaSanITC-Book"}\NormalTok{),}
    \DataTypeTok{plot.title =} \KeywordTok{element_text}\NormalTok{(}\DataTypeTok{family=}\StringTok{"OfficinaSanITC-Book"}\NormalTok{))}
\NormalTok{p11}
\end{Highlighting}
\end{Shaded}

\begin{center}\includegraphics{11_Linear_Regression_Plot_pdf/lr_16-1} \end{center}

\subsection{\texorpdfstring{Using `Five Thirty Eight'
theme}{Using Five Thirty Eight theme}}\label{using-five-thirty-eight-theme}

Below we've applied \texttt{theme\_fivethirtyeight()}, which
approximates graphs in the nice
\href{http://fivethirtyeight.com/}{FiveThirtyEight} website. Again, it
is also important that the font change is optional and it's only to
obtain a more similar result compared to the original. For an exact
result you need `Atlas Grotesk' and `Decima Mono Pro' which are
commercial font and are available
\href{https://commercialtype.com/catalog/atlas}{here} and
\href{https://www.myfonts.com/fonts/tipografiaramis/decima-mono-pro/}{here}.

\begin{Shaded}
\begin{Highlighting}[]
\NormalTok{p11 <-}\StringTok{ }\KeywordTok{ggplot}\NormalTok{(hsi.ckh.returns, }\KeywordTok{aes}\NormalTok{(}\DataTypeTok{x=}\NormalTok{hsi.Risk.premium, }\DataTypeTok{y=}\NormalTok{ckh.Return)) +}\StringTok{ }\KeywordTok{geom_point}\NormalTok{(}\DataTypeTok{shape=}\DecValTok{1}\NormalTok{) +}\StringTok{ }\KeywordTok{geom_smooth}\NormalTok{(}\DataTypeTok{method=}\NormalTok{lm, }\DataTypeTok{se=}\OtherTok{FALSE}\NormalTok{) +}
\StringTok{  }\KeywordTok{ggtitle}\NormalTok{(}\StringTok{"CKH regression line"}\NormalTok{) +}
\StringTok{  }\KeywordTok{scale_x_continuous}\NormalTok{(}\DataTypeTok{name =} \StringTok{"HSI risk premium"}\NormalTok{) +}
\StringTok{  }\KeywordTok{scale_y_continuous}\NormalTok{(}\DataTypeTok{name =} \StringTok{"CKH return"}\NormalTok{) +}
\StringTok{  }\KeywordTok{annotate}\NormalTok{(}\StringTok{"rect"}\NormalTok{, }\DataTypeTok{xmin =} \NormalTok{-}\FloatTok{0.004}\NormalTok{, }\DataTypeTok{xmax =} \FloatTok{0.104}\NormalTok{, }\DataTypeTok{ymin =} \NormalTok{-}\FloatTok{0.056}\NormalTok{, }\DataTypeTok{ymax =} \NormalTok{-}\FloatTok{0.044}\NormalTok{, }\DataTypeTok{fill=}\StringTok{"white"}\NormalTok{, }
    \DataTypeTok{colour=}\StringTok{"red"}\NormalTok{) +}\StringTok{ }
\StringTok{  }\KeywordTok{annotate}\NormalTok{(}\StringTok{"text"}\NormalTok{, }\DataTypeTok{x =} \FloatTok{0.05}\NormalTok{, }\DataTypeTok{y =} \NormalTok{-}\FloatTok{0.05}\NormalTok{, }\DataTypeTok{label =} \KeywordTok{equation}\NormalTok{(fit), }\DataTypeTok{parse =} \OtherTok{TRUE}\NormalTok{) +}\StringTok{ }
\StringTok{  }\KeywordTok{theme_fivethirtyeight}\NormalTok{() +}\StringTok{ }\KeywordTok{scale_fill_fivethirtyeight}\NormalTok{() +}\StringTok{   }
\StringTok{  }\KeywordTok{theme}\NormalTok{(}\DataTypeTok{axis.title =} \KeywordTok{element_text}\NormalTok{(}\DataTypeTok{family=}\StringTok{"Atlas Grotesk Regular"}\NormalTok{),}
    \DataTypeTok{legend.position=}\StringTok{"bottom"}\NormalTok{, }
    \DataTypeTok{legend.direction=}\StringTok{"horizontal"}\NormalTok{,}
    \DataTypeTok{legend.box =} \StringTok{"horizontal"}\NormalTok{, }
    \DataTypeTok{legend.title=}\KeywordTok{element_text}\NormalTok{(}\DataTypeTok{family=}\StringTok{"Atlas Grotesk Regular"}\NormalTok{, }\DataTypeTok{size =} \DecValTok{10}\NormalTok{),}
    \DataTypeTok{legend.text=}\KeywordTok{element_text}\NormalTok{(}\DataTypeTok{family=}\StringTok{"Atlas Grotesk Regular"}\NormalTok{, }\DataTypeTok{size =} \DecValTok{10}\NormalTok{),}
    \DataTypeTok{plot.title=}\KeywordTok{element_text}\NormalTok{(}\DataTypeTok{family=}\StringTok{"Atlas Grotesk Medium"}\NormalTok{), }
    \DataTypeTok{text=}\KeywordTok{element_text}\NormalTok{(}\DataTypeTok{family=}\StringTok{"DecimaMonoPro"}\NormalTok{))}
\NormalTok{p11}
\end{Highlighting}
\end{Shaded}

\begin{center}\includegraphics{11_Linear_Regression_Plot_pdf/lr_17-1} \end{center}

\subsection{Creating your own theme}\label{creating-your-own-theme}

As before, you can modify your plots a lot as \texttt{ggplot2} allows
many customisations. Here is a custom plot where we have modified the
axes, background and font.

\begin{Shaded}
\begin{Highlighting}[]
\NormalTok{p11 <-}\StringTok{ }\KeywordTok{ggplot}\NormalTok{(hsi.ckh.returns, }\KeywordTok{aes}\NormalTok{(}\DataTypeTok{x=}\NormalTok{hsi.Risk.premium, }\DataTypeTok{y=}\NormalTok{ckh.Return)) +}\StringTok{ }\KeywordTok{geom_point}\NormalTok{(}\DataTypeTok{shape=}\DecValTok{1}\NormalTok{) +}\StringTok{ }\KeywordTok{geom_smooth}\NormalTok{(}\DataTypeTok{method=}\NormalTok{lm, }\DataTypeTok{se=}\OtherTok{FALSE}\NormalTok{) +}
\StringTok{  }\KeywordTok{ggtitle}\NormalTok{(}\StringTok{"CKH regression line"}\NormalTok{) +}
\StringTok{  }\KeywordTok{scale_x_continuous}\NormalTok{(}\DataTypeTok{name =} \StringTok{"HSI risk premium"}\NormalTok{) +}
\StringTok{  }\KeywordTok{scale_y_continuous}\NormalTok{(}\DataTypeTok{name =} \StringTok{"CKH return"}\NormalTok{) +}
\StringTok{  }\KeywordTok{annotate}\NormalTok{(}\StringTok{"rect"}\NormalTok{, }\DataTypeTok{xmin =} \FloatTok{0.00}\NormalTok{, }\DataTypeTok{xmax =} \FloatTok{0.1}\NormalTok{, }\DataTypeTok{ymin =} \NormalTok{-}\FloatTok{0.056}\NormalTok{, }\DataTypeTok{ymax =} \NormalTok{-}\FloatTok{0.044}\NormalTok{, }\DataTypeTok{fill=}\StringTok{"white"}\NormalTok{, }\DataTypeTok{colour=}\StringTok{"red"}\NormalTok{) +}\StringTok{ }
\StringTok{  }\KeywordTok{annotate}\NormalTok{(}\StringTok{"text"}\NormalTok{, }\DataTypeTok{x =} \FloatTok{0.05}\NormalTok{, }\DataTypeTok{y =} \NormalTok{-}\FloatTok{0.05}\NormalTok{, }\DataTypeTok{label =} \KeywordTok{equation}\NormalTok{(fit), }\DataTypeTok{parse =} \OtherTok{TRUE}\NormalTok{) +}\StringTok{ }
\StringTok{  }\KeywordTok{theme}\NormalTok{(}\DataTypeTok{panel.border =} \KeywordTok{element_rect}\NormalTok{(}\DataTypeTok{colour =} \StringTok{"black"}\NormalTok{, }\DataTypeTok{fill=}\OtherTok{NA}\NormalTok{, }\DataTypeTok{size=}\NormalTok{.}\DecValTok{5}\NormalTok{),}
    \DataTypeTok{axis.text.x=}\KeywordTok{element_text}\NormalTok{(}\DataTypeTok{colour=}\StringTok{"black"}\NormalTok{, }\DataTypeTok{size =} \DecValTok{9}\NormalTok{), }
    \DataTypeTok{axis.text.y=}\KeywordTok{element_text}\NormalTok{(}\DataTypeTok{colour=}\StringTok{"black"}\NormalTok{, }\DataTypeTok{size =} \DecValTok{9}\NormalTok{), }
    \DataTypeTok{legend.position =} \StringTok{"bottom"}\NormalTok{, }\DataTypeTok{legend.position =} \StringTok{"horizontal"}\NormalTok{,}
    \DataTypeTok{panel.grid.major =} \KeywordTok{element_line}\NormalTok{(}\DataTypeTok{colour =} \StringTok{"#d3d3d3"}\NormalTok{), }
    \DataTypeTok{panel.grid.minor =} \KeywordTok{element_blank}\NormalTok{(), }
    \DataTypeTok{panel.border =} \KeywordTok{element_blank}\NormalTok{(), }\DataTypeTok{panel.background =} \KeywordTok{element_blank}\NormalTok{(),}
    \DataTypeTok{plot.title =} \KeywordTok{element_text}\NormalTok{(}\DataTypeTok{size =} \DecValTok{14}\NormalTok{, }\DataTypeTok{family =} \StringTok{"Tahoma"}\NormalTok{, }\DataTypeTok{face =} \StringTok{"bold"}\NormalTok{),}
    \DataTypeTok{text=}\KeywordTok{element_text}\NormalTok{(}\DataTypeTok{family=}\StringTok{"Tahoma"}\NormalTok{))}
\NormalTok{p11}
\end{Highlighting}
\end{Shaded}

\begin{center}\includegraphics{11_Linear_Regression_Plot_pdf/lr_18-1} \end{center}

\section{Regression diagnostics
plots}\label{regression-diagnostics-plots}

\subsection{Basic diagnostics plots}\label{basic-diagnostics-plots}

An important part of creating regression models is evaluating how well
they fit the data. We can use the package \texttt{ggfortify} to let
\texttt{ggplot2} interpret \texttt{lm} objects and create diagnostic
plots.

\begin{Shaded}
\begin{Highlighting}[]
\KeywordTok{autoplot}\NormalTok{(fit, }\DataTypeTok{label.size =} \DecValTok{3}\NormalTok{)}
\end{Highlighting}
\end{Shaded}

\begin{center}\includegraphics{11_Linear_Regression_Plot_pdf/lr_19-1} \end{center}

\subsection{Using the white theme}\label{using-the-white-theme-1}

We can also customise the appearance of our diagnostic plots. Let's
first use the white theme by again adding \texttt{theme\_bw()}.

\begin{Shaded}
\begin{Highlighting}[]
\KeywordTok{autoplot}\NormalTok{(fit, }\DataTypeTok{label.size =} \DecValTok{3}\NormalTok{) +}\StringTok{ }\KeywordTok{theme_bw}\NormalTok{()}
\end{Highlighting}
\end{Shaded}

\begin{center}\includegraphics{11_Linear_Regression_Plot_pdf/lr_20-1} \end{center}

\subsection{Creating an XKCD style
chart}\label{creating-an-xkcd-style-chart-1}

We can of course apply our other themes as well. Let's try the XKCD
theme.

\begin{Shaded}
\begin{Highlighting}[]
\KeywordTok{autoplot}\NormalTok{(fit, }\DataTypeTok{label.size =} \DecValTok{3}\NormalTok{) +}\StringTok{ }\KeywordTok{theme}\NormalTok{(}\DataTypeTok{panel.border =} \KeywordTok{element_rect}\NormalTok{(}\DataTypeTok{colour =} \StringTok{"black"}\NormalTok{, }\DataTypeTok{fill=}\OtherTok{NA}\NormalTok{, }\DataTypeTok{size=}\NormalTok{.}\DecValTok{5}\NormalTok{),}
    \DataTypeTok{axis.text.x=}\KeywordTok{element_text}\NormalTok{(}\DataTypeTok{colour=}\StringTok{"black"}\NormalTok{, }\DataTypeTok{size =} \DecValTok{9}\NormalTok{), }
    \DataTypeTok{axis.text.y=}\KeywordTok{element_text}\NormalTok{(}\DataTypeTok{colour=}\StringTok{"black"}\NormalTok{, }\DataTypeTok{size =} \DecValTok{9}\NormalTok{),}
    \DataTypeTok{panel.grid.major =} \KeywordTok{element_line}\NormalTok{(}\DataTypeTok{colour =} \StringTok{"#d3d3d3"}\NormalTok{), }
    \DataTypeTok{panel.grid.minor =} \KeywordTok{element_blank}\NormalTok{(), }
    \DataTypeTok{panel.border =} \KeywordTok{element_blank}\NormalTok{(), }\DataTypeTok{panel.background =} \KeywordTok{element_blank}\NormalTok{(),}
    \DataTypeTok{plot.title =} \KeywordTok{element_text}\NormalTok{(}\DataTypeTok{family =} \StringTok{"xkcd-Regular"}\NormalTok{),}
    \DataTypeTok{text=}\KeywordTok{element_text}\NormalTok{(}\DataTypeTok{family=}\StringTok{"xkcd-Regular"}\NormalTok{))}
\end{Highlighting}
\end{Shaded}

\begin{center}\includegraphics{11_Linear_Regression_Plot_pdf/lr_21-1} \end{center}

\subsection{\texorpdfstring{Using `The Economist'
theme}{Using The Economist theme}}\label{using-the-economist-theme-1}

And now the Economist theme.

\begin{Shaded}
\begin{Highlighting}[]
\KeywordTok{autoplot}\NormalTok{(fit, }\DataTypeTok{label.size =} \DecValTok{3}\NormalTok{) +}\StringTok{ }\KeywordTok{theme_economist}\NormalTok{() +}
\StringTok{  }\KeywordTok{theme}\NormalTok{(}\DataTypeTok{panel.border =} \KeywordTok{element_rect}\NormalTok{(}\DataTypeTok{colour =} \StringTok{"black"}\NormalTok{, }\DataTypeTok{fill=}\OtherTok{NA}\NormalTok{, }\DataTypeTok{size=}\NormalTok{.}\DecValTok{5}\NormalTok{),}
    \DataTypeTok{axis.text.x=}\KeywordTok{element_text}\NormalTok{(}\DataTypeTok{colour=}\StringTok{"black"}\NormalTok{, }\DataTypeTok{size =} \DecValTok{9}\NormalTok{), }
    \DataTypeTok{axis.text.y=}\KeywordTok{element_text}\NormalTok{(}\DataTypeTok{colour=}\StringTok{"black"}\NormalTok{, }\DataTypeTok{size =} \DecValTok{9}\NormalTok{),}
    \DataTypeTok{panel.border =} \KeywordTok{element_blank}\NormalTok{(), }\DataTypeTok{panel.background =} \KeywordTok{element_blank}\NormalTok{(),}
    \DataTypeTok{plot.title =} \KeywordTok{element_text}\NormalTok{(}\DataTypeTok{family =} \StringTok{"OfficinaSanITC-Book"}\NormalTok{),}
    \DataTypeTok{text=}\KeywordTok{element_text}\NormalTok{(}\DataTypeTok{family=}\StringTok{"OfficinaSanITC-Book"}\NormalTok{))}
\end{Highlighting}
\end{Shaded}

\begin{center}\includegraphics{11_Linear_Regression_Plot_pdf/lr_22-1} \end{center}

\subsection{\texorpdfstring{Using `Five Thirty Eight'
theme}{Using Five Thirty Eight theme}}\label{using-five-thirty-eight-theme-1}

And now Five Thirty Eight theme.

\begin{Shaded}
\begin{Highlighting}[]
\KeywordTok{autoplot}\NormalTok{(fit, }\DataTypeTok{label.size =} \DecValTok{3}\NormalTok{) +}\StringTok{ }\KeywordTok{theme_fivethirtyeight}\NormalTok{() +}\StringTok{ }
\StringTok{  }\KeywordTok{theme}\NormalTok{(}\DataTypeTok{axis.title =} \KeywordTok{element_text}\NormalTok{(}\DataTypeTok{family=}\StringTok{"Atlas Grotesk Regular"}\NormalTok{),}
    \DataTypeTok{legend.position=}\StringTok{"bottom"}\NormalTok{, }
    \DataTypeTok{legend.direction=}\StringTok{"horizontal"}\NormalTok{,}
    \DataTypeTok{legend.box =} \StringTok{"horizontal"}\NormalTok{,}
    \DataTypeTok{plot.title=}\KeywordTok{element_text}\NormalTok{(}\DataTypeTok{family=}\StringTok{"Atlas Grotesk Medium"}\NormalTok{, }\DataTypeTok{size =} \DecValTok{14}\NormalTok{), }
    \DataTypeTok{text=}\KeywordTok{element_text}\NormalTok{(}\DataTypeTok{family=}\StringTok{"DecimaMonoPro"}\NormalTok{))}
\end{Highlighting}
\end{Shaded}

\begin{center}\includegraphics{11_Linear_Regression_Plot_pdf/lr_23-1} \end{center}

\subsection{Creating your own theme}\label{creating-your-own-theme-1}

Finally, we can also fully customise the diagnostic plots to match our
regression plot simply by applying all of the same theme options.

\begin{Shaded}
\begin{Highlighting}[]
\KeywordTok{autoplot}\NormalTok{(fit, }\DataTypeTok{label.size =} \DecValTok{3}\NormalTok{) +}\StringTok{ }\KeywordTok{theme}\NormalTok{(}\DataTypeTok{panel.border =} \KeywordTok{element_rect}\NormalTok{(}\DataTypeTok{colour =} \StringTok{"black"}\NormalTok{, }\DataTypeTok{fill=}\OtherTok{NA}\NormalTok{, }\DataTypeTok{size=}\NormalTok{.}\DecValTok{5}\NormalTok{),}
    \DataTypeTok{axis.text.x=}\KeywordTok{element_text}\NormalTok{(}\DataTypeTok{colour=}\StringTok{"black"}\NormalTok{, }\DataTypeTok{size =} \DecValTok{9}\NormalTok{), }
    \DataTypeTok{axis.text.y=}\KeywordTok{element_text}\NormalTok{(}\DataTypeTok{colour=}\StringTok{"black"}\NormalTok{, }\DataTypeTok{size =} \DecValTok{9}\NormalTok{), }
    \DataTypeTok{legend.position =} \StringTok{"bottom"}\NormalTok{, }\DataTypeTok{legend.position =} \StringTok{"horizontal"}\NormalTok{,}
    \DataTypeTok{panel.grid.major =} \KeywordTok{element_line}\NormalTok{(}\DataTypeTok{colour =} \StringTok{"#d3d3d3"}\NormalTok{), }
    \DataTypeTok{panel.grid.minor =} \KeywordTok{element_blank}\NormalTok{(), }
    \DataTypeTok{panel.border =} \KeywordTok{element_blank}\NormalTok{(), }\DataTypeTok{panel.background =} \KeywordTok{element_blank}\NormalTok{(),}
    \DataTypeTok{plot.title =} \KeywordTok{element_text}\NormalTok{(}\DataTypeTok{size =} \DecValTok{14}\NormalTok{, }\DataTypeTok{family =} \StringTok{"Tahoma"}\NormalTok{, }\DataTypeTok{face =} \StringTok{"bold"}\NormalTok{),}
    \DataTypeTok{text=}\KeywordTok{element_text}\NormalTok{(}\DataTypeTok{family=}\StringTok{"Tahoma"}\NormalTok{))}
\end{Highlighting}
\end{Shaded}

\begin{center}\includegraphics{11_Linear_Regression_Plot_pdf/lr_24-1} \end{center}

\end{document}
